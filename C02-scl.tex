\chapter{The Case of Smart City Learning}
\label{cha:smart-cities}

In this chapter, I introduce the domain of \textit{SCL}, used as a case and an application domain. SCL was employed as a strategy to promote sustainability, awareness and lifelong learning about the challenges affecting modern smart cities. It was used as a design goal, proposed by researchers or facilitators, but directly pursued by the users involved in the studies.

In the remaining of the thesis, the technical solutions described in Chapter~\ref{cha:iot-framework} and theoretical frameworks presented in Chapter~\ref{cha:theory} complement the SCL vision by describing how practical applications can be deployed in the field.


\section{Smart Cities}

Nowadays, most of the world's population live in cities. While cities occupy only 2\% of the surface of the world, they are responsible for 75\% of the global energy consumption and 80\% of $CO_2$ emissions. The concept of smart cities arose as a means to foster innovation and technology adoption and develop data networks in urban environments. Smart city applications focus on a wide range of sub-domains like governance and service infrastructure (i.e. smart grids and water management) or on collecting and utilising data. Smart cities are founded on communities of citizens; the social aspect plays an important role in ensuring a meaningful deployment of the technology, which should serve the citizens despite their diversities. 

One of the most cited definitions in this regard is the one advanced by Caragliu, Del Bo and Nijkamp \autocite*{caragliu_smart_2011}, who stated that a city is smart \enquote{\textit{when investments in human and social capital and traditional (transport) and modern (ICT) communication infrastructure fuel sustainable economic growth and a high quality of life, with a wise management of natural resources, through participatory governance}}.

ICT has recently become part of the mainstream debate on urban sustainability as well as urbanisation because of the ubiquity presence of urban computing and the massive use of urban ICT in urban systems and domains \autocite{bibri_smart_2017}. Indeed, data sensing and information processing are being fast embedded into the very fabric of contemporary cities \autocites{batty_smart_2012}{bibri_big_2016}. A large number of advanced technologies are being developed and applied in response to the urgent need for dealing with the complexity of the knowledge necessary for enhancing, harnessing and integrating urban systems and facilitating collaboration and coordination among urban domains in the realm of smart sustainable urban planning and development \autocite{bibri_big_2016}.

The emergence of this new techno-urban phenomenon has been particularly fuelled by what is labelled \enquote{ICT of the new wave of computing}, that is, a combination of various forms of pervasive computing, the most prevalent of which are ubiquitous computing, ambient intelligence (AmI), IoT and sentient computing \autocite{bibri_social_2017}.

In my work, I envision a smart city firstly as a community of citizens that live and move in the urban environment, as part of their daily routine. This vision finds its roots in the work of Hollands \autocite*{hollands_will_2008}, aiming to support regional competitiveness. Technology is seen as an enabling factor, which does not disrupt the activities of the citizens but rather encourages awareness, builds problem-solving skills and supports lifelong learning, towards an improved quality of life.


\section{Citizen Participation}

Despite the volume of research targeting smart cities, citizens, especially young ones, are often included only for symbolic purposes; meaningful inclusion remains an open challenge \autocite{lansdown_realisation_2010}. While conventional methods of public participation like committee groups and public hearings have failed to engage the majority of the public \autocites{roberts_public_2004}{irvin_citizen_2004}, multidisciplinary methods are currently being investigated in order to give voice to citizens and stakeholders through authentic dialogue, building social capital and trust \autocite{innes_reframing_2004}.

Previous research has voiced the need for ICT development and innovation to be linked with sustainable development and, thus, related future investment to be justified by environmental concerns and socio–economic needs, rather than technical advancement and industrial competitiveness \autocite{bibri_smart_2017}.

In essence, there are two mainstream approaches to smart cities: (i) the technology- and ICT–oriented approach and (ii) the people–oriented approach. Specifically, there are smart city strategies that focus on the efficiency and advancement of hard infrastructure and technology (transport, energy, communication, waste, water, etc) through ICT and strategies that focus on the soft infrastructure and people (social and human capital in terms of knowledge, participation, equity, safety, etc) \autocite{angelidou_smart_2014}. Several challenges have been identified, among which to explore the notion of the city as a laboratory for innovation and to develop technologies that ensure informed participation and create shared knowledge for democratic city governance \autocite{batty_smart_2012}. As for the second approach, Neirotti et al. \autocite*{neirotti_current_2014} described smart cities as a means of enhancing the quality of life of citizens. Smart cities entail human and social factors, apart from physical and technological factors \autocite{galan-garcia_accelerated-time_2014}. Lombardi et al. \autocite*{lombardi_advanced_2012} emphasised additional soft factors such as participation, safety and cultural heritage.

Local city governments are currently investing in advanced ICT to provide technological infrastructures supporting AmI and UbiComp, as well as to foster respect for the environmental and social responsibility \autocite{solanas_smart_2014}. In order to facilitate participation and inclusion under these premises, situated engagement might help by increasing the awareness of the participants regarding the challenges and opportunities of the urban environment where they live and work.

\section{Learning}

The process aiding citizen participation is also a process of learning, improving environmental awareness, knowledge and personal skills \autocite{wilks_voice_2013} and teaching people how to negotiate and respect each other's views \autocite{corsi_child_2002}. 

Smart cities are a recognised eco-system for learning. SCL aims to support the improvement of all key factors contributing to the regional competitiveness of cities: mobility, environment, people, quality of life and governance \autocite{hollands_will_2008}. This approach aspires at optimising resource consumption and saving time, improving flows of people, goods and data\footnote{www.mifav.uniroma2.it/inevent/events/sclo/}.
Data produced in the city, social challenges affecting the population and technologies are all taken into account and used to increase awareness and lifelong learning through technological applications where citizens represent active actors.


\section{Technology}

Technology is a fundamental component of smart cities. Through technology, urban data can be collected, aggregated and analysed to extract valuable information. Digital solutions can also help lower the costs and increase the efficiency in many strategic sectors like governance, transportation, infrastructure and social services for the citizens. Urban data can support researchers and decision-makers in discovering the hidden layers of smart cities, highlighting patterns and processes that were otherwise impossible to detect \autocite{vazifeh_addressing_2018}.

IoT plays an important role as an enabling technology for this vision. Sensor networks and M2M systems are largely employed to monitor the city and sense and collect the data generated in the urban environment. In addition to this data-driven approach, ongoing research is also exploring how technology can be used to better connect with the population, reinforcing the social aspect and building a sense of community.


\section{SCL as a Domain Case}

The philosophy of SCL and the values it promotes present many points of contact with the social and technological framing described in Chapter~\ref{cha:introduction}. More in particular, SCL is a relevant domain case for my work since it is connected to the following notions:

\begin{itemize}
	\item Citizens as active actors, directly involved in and contributing to the urban environment, instead of being relegated to a passive role.
	\item The urban population as non-expert users, citizens often fit in this category, which represents the target user group for my research.
	\item The paramount role of new technologies, able to keep the user in the loop.
	\item The use of technology as a tool to also promote learning, inclusion and social awareness.
	\item The possibility of creating meaningful technological applications, tackling recognised challenges affecting most of the population.
\end{itemize}

These shared values create both an opportunity to contribute to the SCL domain, by generating new solutions and methods addressing societal challenges affecting cities, and a compelling design space to experiment with innovative technologies.

\chapter{Evaluation}
\label{cha:evaluation}

This chapter describes the evaluation of the contributions presented in Chapter~\ref{cha:contributions} in connection with the research questions. In addition, limitations are discussed, and the external impact of the research work is summarised.


\section{Evaluation of Research Questions}

\subsection*{MRQ: \MRQ}
\label{mrq}
The main research question is answered by Contributions 1, 2 and 3. Using co-design workshops, it was demonstrated how concepts from theories in tangible user interaction can be deployed into a suite of tools, applications, methods and artefacts. This integrated suite was demonstrated to be useful and effective in supporting non-experts in generating ideas and prototypes of tangible interfaces and smart augmented objects. Different tools addressed different phases of the process, whereas a holistic framework was designed to support the users throughout the whole journey.

\subsection*{RQ1: \RQi}
\label{rq1}
The answer to the first research question is provided by Contribution 1. The work included in C1 describes in detail the tools and co-design methods employed during the phases included in the process defined by the Tiles IoT framework. Emphasis is placed on how the transition between the idea and the prototype is facilitated, maintaining a complexity level suitable for non-experts.

\subsection*{RQ2: \RQii}
\label{rq2}
This research question is answered by Contributions 2 and 4. C2 includes the studies where the Tiles ideation toolkit was firstly introduced and the ones where it was specialised in the domains of smart cities and IoT applications for reflective learning. C4 explores how the toolkits and the methods developed can be employed to support SCL.

\subsection*{RQ3: \RQiii}
\label{rq3}
This research question is answered by Contribution 3 and 5. A technical architecture to support prototyping of IoT applications is described in C5. Its implementation into a rapid prototyping toolkit is then proposed and evaluated during prototyping workshops with non-expert users. C3 focuses on these phases of programming, prototyping and tangible exploration of ideas.


\section{Limitations of the Research Approach}

The limitations of the research approach can be divided into factors connected to the nature of the case studies performed and limitations related to the role of the learning methods mentioned in the thesis.

\subsection{Case Studies}
The limited time at our disposal during some of the case studies often prevented to perform personal interviews with the users. We were able to perform interviews during the first half of the studies, however when the evaluation workshops started to involve larger groups of users, limited time and resources prevented any case-by-case in depth assessment with the users.

The diverse conditions in which the studies took place affected users and researchers. When generalising the findings cross-case, conditions for each study had to be taken into account in order to provide a robust interpretation of the phenomenon. This heterogeneity added complexity and limited the possibilities of generalisation of the results.

Long-term effects were difficult to assess for the same reasons described above. Especially for the prototyping phase, it was not possible to assess more than a short prototyping and programming iteration, while long-term dynamics connected to an iterative prototyping process might have led to additional understandings.

Given the different conditions in which the workshops took place and the human factors involved, it was challenging to replicate the same protocol in all the studies. Human factors also led to unforeseen situations that had to be addressed on site, quickly and case-by-case.

\subsection{Role of Learning}
Learning and related practices are recalled and addressed in several forms within the work described in this thesis.
For example (i) learning is part of SCL, the application domain used during the workshops, (ii) the learning outcome of the users in terms of acquired knowledge about IoT and other topics associated with the workshop activities were evaluated during some of the studies, (iii) an extension of the Tiles ideation toolkit for IoT application for reflective learning was created and (iv) a group of cards included in the Tiles ideation toolkit encouraged the users to reflect on their idea analysing it under a given perspective.

The toolkits developed were not originally designed as learning tools, but their potential in facilitating learning emerged during the evaluation studies. For this reason, I decided together with my research group to include an assessment of the learning outcome in the evaluation. However, learning is not part of my research questions, in fact assessing the learning outcome was not the main objective of the case studies performed.

Learning as part of SCL was not evaluated since SCL was used mainly as an application domain, to provide the users a design space for their IoT ideas. The ultimate goal of the studies was not to create an IoT product to solve the challenges connected to SCL, since this would have required an evaluation of the fully developed ideas deployed on field, a task out of the scope of my PhD.

The extension of the Tiles ideation toolkit covering IoT applications for reflective learning (P4) contributed to demonstrate how the toolkit could be specialised to target specific goals and application domains (RQ2). For the same reasons described above, the efficacy of the ideas was not tested on the field, however we evaluated if the necessary characteristics of an application for reflective learning were included in the concept.

Finally, the \textit{criteria} cards of the Tiles ideation toolkit invite the users to reflect and adapt their idea during the design process, proposing some reflection lenses. However, this design phase cannot be considered as a structured reflective learning process, given that for example it could simply consist in adapting the idea to increase its market potential or feasibility.


\section{Exploitation and External Impact of Research}
\label{sec:exploitation}

The Tiles ideation toolkit has been largely and independently employed in the research, governance, industry and education sectors from institutions in Europe, America, Asia and Australia.
A few examples are the works of 
Gennari et al. \autocite*{gennari_design_2017}, Sintoris at al. \autocite*{sintoris_out_2018}, Avouris et al. \autocite*{avouris_designing_2018} and  Zhai et al. \autocite*{zhai_co-sleep_2018}. The Tiles ideation toolkit was also selected by an independent group of researchers to support an ideation and rapid prototyping workshop during the CHI Conference on Human Factors in Computing Systems, in 2018 \autocite{angelini_internet_2018-1}.

The Tiles ideation toolkit was employed at NTNU during workshops as part of the master-level university courses in \textit{Cooperation Technology and Social Media} (TDT4245) at the Department of Computer Science, \textit{Prototyping Interactive Media} (TPD4126) at the Department of Design and \textit{Design of Communicating Systems} (TTM4115) at the Department of Information Security and Communication Technology.

A small community formed around the Tiles project, with particular emphasis on the Tiles ideation toolkit, but with high interest also in the tools for programming and rapid prototyping. Around 10 groups approached us directly to ask for guidance, customisation and assistance in running the workshop. Thanks to the open-source licence of the cards, cardboard and workshop technique, several contributed to creating extensions for different domains, like smart buildings, or creating typesets to ease printing and collaboration tasks.

A company was started to refine and integrate the prototypes produced during the research activities, with the objective of developing a kit suitable for distribution.
A Kickstarter campaign to crowdfund the first production batch of the Tiles ideation toolkit is currently undergoing\footnote{https://www.kickstarter.com/projects/tiles/tiles-iot-inventor-toolkit}.

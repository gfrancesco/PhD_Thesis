\chapter{Evaluation}
\label{cha:evaluation}

This chapter describes the evaluation of the contributions presented in Chapter~\ref{cha:contributions} in connection with the research questions. In addition, validity threats are discussed, and the external impact of the research work is summarised.


\section{Evaluation of Research Questions}

\subsection*{MRQ: \MRQ}
\label{mrq}
The main research question is answered by Contributions 1, 2 and 3. Using co-design workshops, it was demonstrated how concepts from theories in tangible user interaction can be deployed into a suite of tools, applications, methods and artefacts. This integrated suite was demonstrated to be useful and effective in supporting non-experts in generating ideas and prototypes of tangible interfaces and smart augmented objects. Different tools addressed different phases of the process, whereas a holistic framework was designed to support the users throughout the whole journey.

\subsection*{RQ1: \RQi}
\label{rq1}
The answer to the first research question is provided by Contribution 1. The work included in C1 describes in detail the tools and co-design methods employed during the phases included in the process defined by the Tiles IoT framework. Emphasis is placed on how the transition between the idea and the prototype is facilitated, maintaining a complexity level suitable for non-experts.

\subsection*{RQ2: \RQii}
\label{rq2}
This research question is answered by Contributions 2 and 4. C2 includes the studies where the Tiles ideation toolkit was firstly introduced and the ones where it was specialised in the domains of smart cities and IoT applications for reflective learning. C4 explores how the toolkits and the methods developed can be employed to support SCL.

\subsection*{RQ3: \RQiii}
\label{rq3}
This research question is answered by Contribution 3 and 5. A technical architecture to support prototyping of IoT applications is described in C5. Its implementation into a rapid prototyping toolkit is then proposed and evaluated during prototyping workshops with non-expert users. C3 focuses on these phases of programming, prototyping and tangible exploration of ideas.


\section{Evaluation of the Research Approach in Field Studies}
\label{evaluation-of-research-approach}

The research approach employed during the investigation described in this thesis is evaluated here. Validity and reliability issues \autocites{yin_case_2017}{riege_validity_2003} are also discussed.

Criteria were established for judging the quality of the case study method \autocite{riege_validity_2003}. Several tests and techniques were synthesised for establishing validity and reliability in case study research.
In the following section, the research approach is evaluated following the guidelines and design tests reported by Riege \autocite*{riege_validity_2003}; however, not all the tests devised by Riege apply to the studies included in this thesis.

\subsection{Construct Validity}
\label{construct-validity}
Construct validity evaluates whether appropriate operational measures have been adopted for the theoretical concepts being researched.
Collecting data using multiple data sources increases construct validity and protects against researcher bias \autocites{perakyla_reliability_1998}{flick_triangulation_1992}. Converging findings emerged when analysing different sources through triangulation. This has been the case for the data collected during the evaluation workshops. We collected quantitative and qualitative data through questionnaires, structured interviews, observations, video and audio recordings and analysis of the artefacts produced. Chains of evidence in the data \autocites{griggs_analysing_1987}{hirschman_humanistic_1986} were highlighted when summarising the outcome of the data analysis process.

The number of participants involved in the studies added up to more than 500. A fraction of them were involved in the studies presented in this thesis. Significant experience was gathered in the process, such experience contributed to the definition of theories and in providing insights while analysing qualitative and quantitative data.
The size of the data sample allowed for some statistical analysis, but I looked for confirmation in the qualitative data before formalising the results. Given the number of human factors involved, this strategy was demonstrated to be more robust and allowed for better interpretation of the data collected.

During the studies, researchers had close and direct personal contact with the organisations and users involved. Effort was then made to refrain from subjective judgements during the periods of research design and data collection, to enhance construct validity.

Each of the studies was limited to one or two sessions, lasting usually two to three hours each. It was, therefore, not possible to prove long-term effects for the tools developed.

\subsection{Internal Validity}
Internal validity, as traditionally known in quantitative research, refers to the establishment of cause-and-effect relationships, while the emphasis on constructing an internally valid research process in case study research lies in establishing phenomena in a credible way \autocite{riege_validity_2003}. Researchers should not only highlight major patterns of similarities and differences between the experience of the respondents and their beliefs but also try to identify what components are significant for those examined patterns and what mechanisms produced them.

Data from the field studies included in the papers presented in this thesis were checked cross-case to assess the internal coherence of findings \autocite{miles_qualitative_1994}. Illustrations and diagrams eased this task, allowing the identification and evaluation of evidence, cross-checking within-case and cross-case \autocite{yin_case_2017}.

During field studies, it was sometimes required to deviate from the agreed protocol because of unpredictable events. In addition, it was difficult to replicate the same exact conditions because of the human factors involved and the lack of control over some of the variables. Workshops in the classroom were often limited in time, involved a variable number of students and took place at different times of the day.

Despite the challenges, I was able to gather enough evidence to complete the design work. The experience matured was helpful in extracting valuable know-how from noisy data sets.

\subsection{External Validity}
External validity is concerned with the extrapolation of particular research findings beyond the immediate form of inquiry to the general.

While quantitative research, for example, using surveys aims at statistical generalisation and synthesis as methods to pursue external validity, case studies rely on analytic generalisation, whereby particular findings are generalised to some broader theory. The focus lies on an understanding and exploration of constructs, that is, usually the comparison of initially identified and/or developed theoretical constructs and the empirical results of single or multiple case studies \autocite{riege_validity_2003}.

In order to increase the external validity, several techniques were employed. The logic of the case study was replicated across different domains \autocites{eisenhardt_building_1989}{parkhe_messy_1993}: most of the workshops involved students, but the tools were evaluated also with other target users (e.g. in P4), including researchers, employees from the municipality, entrepreneurs, programmers and freelancers.

The boundaries and scope of the research were defined in the research design phase \autocite{marshall_designing_2014}. The outcome for each of the phases covered by the toolkits was clearly defined, as well as the target users of the toolkits and their attributes (see Chapter~\ref{cha:iot-framework} for details).

Lastly, comparison of evidence with the extant literature in the domain of interest helped in clearly outlining the contributions and in generalising, always within the scope and boundaries of the research \autocite{yin_case_2017}.

\subsection{Reliability}
Reliability refers to the demonstration that the operations and procedures of the research inquiry can be repeated by other researchers who then achieve similar findings; that is, the extent of findings can be replicated assuming that, for example, the interviewing techniques and procedures remain consistent \autocite{riege_validity_2003}.

In case study research, this can raise problems as people are not as static as measurements used in quantitative research, and even if researchers were concerned about ensuring that others can precisely follow each step, the results may still differ. Indeed, data on real-life events, which were collected by different researchers, may not converge into one consistent picture. However, possible differences also can provide a valuable additional source of information about the cases investigated \autocite{riege_validity_2003}.

The techniques used to increase reliability included the recording of observations and actions as concretely as possible \autocite{lecompte_problems_1982}, the use of pilot studies to develop and refine the case study protocol \autocites{eisenhardt_building_1989}{mitchell_industrial_1993}{yin_case_2017}, the use of multiple researchers who continually communicate about methodological decisions \autocite{lecompte_problems_1982}, the mechanical recording of data \autocite{nair_using_1995}, the development of a case study database to organise and document the mass of collected data \autocite{lincoln_naturalistic_1985} and finally the use of peer review and examination \autocite{lecompte_problems_1982}.

Repeatability can be demonstrated by the existence of external publications employing the Tiles ideation toolkit in contexts similar to the ones presented in the papers included in this thesis (see Section~\ref{sec:exploitation} for details). In some cases, the ideation process directing the use of the Tiles cards was modified or extended (e.g. removing the cardboard). The results reported, however, are aligned with the findings presented in this thesis.

Relative to repeatability, limitations might affect the prototyping phase. This phase, in fact, was evaluated and tested only internally during the last months of my PhD. External adoption and independent evaluations of the prototyping toolkit are not yet possible because of the early stage of development of the hardware and software stacks.


\section{Exploitation and External Impact of Research}
\label{sec:exploitation}

The Tiles ideation toolkit has been largely and independently employed in the research, governance, industry and education sectors from institutions in Europe, America, Asia and Australia.
A few examples are the works of 
Gennari et al. \autocite*{gennari_design_2017}, Sintoris at al. \autocite*{sintoris_out_2018}, Avouris et al. \autocite*{avouris_designing_2018} and  Zhai et al. \autocite*{zhai_co-sleep_2018}. The Tiles ideation toolkit was also selected by an independent group of researchers to support an ideation and rapid prototyping workshop during the CHI Conference on Human Factors in Computing Systems, in 2018 \autocite{angelini_internet_2018-1}.

The Tiles ideation toolkit was employed at NTNU during workshops as part of the master-level university courses in \textit{Cooperation Technology and Social Media} (TDT4245) at the Department of Computer Science, \textit{Prototyping Interactive Media} (TPD4126) at the Department of Design and \textit{Design of Communicating Systems} (TTM4115) at the Department of Information Security and Communication Technology.

A small community formed around the Tiles project, with particular emphasis on the Tiles ideation toolkit, but with high interest also in the tools for programming and rapid prototyping. Around 10 groups approached us directly to ask for guidance, customisation and assistance in running the workshop. Thanks to the open-source licence of the cards, cardboard and workshop technique, several contributed to creating extensions for different domains, like smart buildings, or creating typesets to ease printing and collaboration tasks.

A company was started to refine and integrate the prototypes produced during the research activities, with the objective of developing a kit suitable for distribution.
A Kickstarter campaign is currently scheduled to crowdfund the first production batch of the Tiles ideation toolkit.

\chapter{Conclusions and Future Work}
\label{cha:conclusions}


The work described in this thesis focused on enabling non-experts to generate ideas and prototypes of IoT applications involving augmented objects and tangible interfaces.

The main research method adopted was design science research. Several user studies were performed to evaluate the tools developed and the creative process supporting the users from idea generation to prototyping.
The work was grounded in the literature about SCL, co-design, tangible interfaces and user-centred design.
Prototypes and tools were produced in the form of a design toolkit, electronic devices for object augmentation, mobile applications and a software toolkit. Each of them was developed through multiple design iterations, producing various intermediate versions at different levels of fidelity.
The scientific work has been published in peer-reviewed journals and conference proceedings, and seven of these publications were included in this thesis.

The research questions were answered by five contributions, hereafter summarised in a set of conclusions that also delineate future work. 


\section*{Conclusion 1}

The experience matured and the tools produced can be used to support non-experts in learning about IoT and smart cities, generating an IoT application idea starting from a real problem and prototyping such idea into a programmable demonstrator.
This emerged as a complex process to define and evaluate, involving a diverse set of skills that require team effort in the areas of learning, social sciences, programming, networking protocols, electronics and embedded hardware and software.
Prototypes covering each of these areas were created and successfully evaluated for their impact on supporting non-experts throughout the whole creative process (P3, P7).
Lessons learned and feedback gathered drove the definition of new theories and guidelines (P7).

Additional work is required to finalise the design of the tools, especially the ones covering the prototyping phase. Given the positive feedback collected, future work will generalise tools and methods to cover other application domains in addition to smart cities.


\section*{Conclusion 2}

The holistic IoT ideation process, spanning from brainstorming to prototyping, already covered in C1, was investigated in two phases. The first phases covered the process of idea generation and design, during which the users familiarised with the application domain and the solution space and produced an IoT application idea while brainstorming in group-based workshops. These activities were supported through the Tiles ideation toolkit, a card-based toolkit developed and evaluated in multiple iterations (P2, P3 and P4).
We received excellent feedback during the user studies, and the toolkit was independently adopted by schools and institutions all over the world. We experimented with extensions targeting specific domains and learning strategies (P4).

Future work might include the internationalisation of the cards and the cardboard, in order to promote adoption in non-English-speaking countries. Minor enhancements in the card decks are also planned to further reduce the complexity and increase the ease of use. Finally, new cards could eventually be created to specialise the toolkit for specific application domains, like healthcare or smart homes.


\section*{Conclusion 3}

The final phases of the process covered in C1 address rapid prototyping and tangible exploration of ideas. These phases were designed to integrate with the idea generation phase and to allow non-experts to produce a prototype of the augmented objects envisioned. The RapIoT toolkit (P6) supported the technical infrastructure employed by the users during programming and prototyping. The technologies include a cloud development environment, an application for mobile devices, a communication protocol and a firmware for embedded devices. The process and the technical solutions were successfully evaluated (P7) through user studies where non-experts developed and programmed prototypes of augmented objects.

Future work will be oriented towards revising the technical tools in order to adopt open standards, increase the efficiency and simplify the programming paradigm and the deployment procedure.


\section*{Conclusion 4}

While running the field studies, the educational potential of the Tiles ideation toolkit naturally emerged. A side research line was then dedicated to exploring more systematically the outcome in terms of learning (P5). When the Tiles ideation toolkit was employed as a learning tool, we received good feedback in terms of improved knowledge about IoT and smart cities. The combination between technology and domain expertise was also beneficial in conveying SCL (P2, P5 and P7).

Future improvements to support the learning outcome can include better integration into the curricula of the schools. Internationalisation of the tools might also benefit the younger students, who occasionally find it difficult to understand the terminology employed without being assisted by the teacher.


\section*{Conclusion 5}

In the process of designing, developing and evaluating the tools and methods described in this thesis, theoretical and technical knowledge about the IoT domain was acquired. Design and technical challenges became evident while experimenting and were later confirmed by reviewing the relevant domain literature (Chapter~\ref{cha:iot-framework}). Based on such literature, a descriptive draft of an IoT framework was redacted. This framework aims on the one hand to overcome the identified challenges and on the other hand to extend the support towards a concept of IoT that includes humans in the loop, standardising tangible interaction primitives and object manipulation. Future work will include a formal definition of the framework and an initial implementation that integrates with the architecture described in C3.

\bigskip
\begin{center}
\noindent\rule{8cm}{0.4pt}    
\end{center}
\bigskip

In conclusion, this thesis has developed knowledge and tools about a holistic process supporting non-experts in learning, ideation and prototyping of IoT applications for augmented objects. My investigation used SCL as a domain case, but the basis for generalising the theories and technologies developed has been settled.
Commercial exploitation is currently being explored, and future work aims to improve the prototypes created and to explore new application domains while maintaining the theoretical foundations that backed the work presented here.

\chapter{Results}
\label{cha:results}

This chapter summarises the papers that document the conducted research.


\section{Overview of the Research Papers}
\label{papers}

Research work was published in peer-reviewed journals and conference proceedings. Seven of these publications are included in this thesis; three journal papers and four conference papers. The articles are summarised in this section, including the following:
\begin{itemize}
	\itemsep1pt\parskip0pt\parsep0pt 
	\item Title 
	\item Authors and their roles in the paper 
	\item Abstract of the paper
	\item Publisher 
	\item A short description of how the paper relates to the research questions
\end{itemize}

Papers are reprinted in full in Part II of the thesis.


\section[P1: Technology-Enhanced Smart City Learning: A Systematic Mapping of the Literature.][Paper 1]{Paper 1}
\label{paper-1}

\emph{Title}: Technology-Enhanced Smart City Learning: A Systematic Mapping of the Literature.

\emph{Authors}: Francesco Gianni and Monica Divitini.

\emph{Contributions of the authors}: Gianni led the research and the paper writing. He was actively involved in programming the study and collecting the data. The screening of the papers was performed mainly by Gianni, while articles coding was performed in equal measure by Gianni and Divitini. Divitini also provided general supervision for the research and the paper writing.

\begin{quote}
	\emph{Abstract:} Smart cities are a popular and recognised research topic. In urban spaces, the learning factor is an important component for citizens and local communities. This paper presents a systematic mapping of the literature on smart city learning, with focus on how technology is used to enhance smart city learning. The goal is to map the state of the art and to identify gaps in current research that can prompt new research in this area. Articles were collected from various online databases and relevant journal publications, selected according to defined inclusion/exclusion criteria. Abstracts were coded based on a number of criteria, including e.g. learning goal, used technology, and theoretical approach. Following the coding process results were analysed to identify themes. In the paper we shed light on the current understanding of smart city learning by (i) identifying common scenarios and learning settings; (ii) publication patterns; (iii) technical features in the supporting technology; (iv) learning theories and approaches that are mostly used; and (v) adopted type of research and research methods. The mapping shows that the concept of smart city learning is growing in popularity, with increasing number of publications in this area in the last years. However, the field is rather fragmented, with very different understanding of the concept.  Smart city learning is also emerging as a very complex form of learning, with different stakeholders, learning activities, and technological solutions combined in rich eco-systems. The mapping also points out two largely unexplored areas of technological support, namely the Internet of Things (IoT) and the use of city-related data.
\end{quote}

\emph{Published in}: Interaction Design and Architecture(s) Journal - IxD\&A 27:28--43, (2016).

\emph{Description}: This paper maps the literature in the smart city learning domain. It provides grounding and identifies gaps in the literature, thus supporting the solutions designed in the subsequent works. The findings shed light on a field where modern technologies are seldom employed, and studies do not involve an heterogeneous sample of the citizens, nor they promote active participation.
The paper started the investigation of RQ2, addressing how modern technologies can be tailored to the smart city learning domain.


\section[P2: Tiles: A Card-Based Ideation Toolkit for the Internet of Things.][Paper 2]{Paper 2}
\label{paper-2}

\emph{Title}: Tiles: A Card-Based Ideation Toolkit for the Internet of Things.

\emph{Authors}: Simone Mora, Francesco Gianni and Monica Divitini.

\emph{Contributions of the authors}: Mora created the toolkit and led the paper writing. Gianni created domain specific personas and scenarios used during the evaluation, analysed the data and wrote the second half of the paper. All the authors participated in the data collection during the evaluation workshops. Divitini provided general supervision for the research and the paper writing.

\begin{quote}
	\emph{Abstract:} The Internet of Things (IoT) offers new opportunities to invent technology-augmented things that are more useful, efficient or playful than their ordinary selves, yet only a few tools currently support ideation for the IoT. In this paper we present Tiles Cards, a set of 110 design cards and a workshop technique to involve non-experts in quick idea generation for augmented objects. Our tool aims to support exploring combinations of user interface metaphors, digital services, and physical objects. Then it supports creative thinking through provocative design goals inspired by human values and desires. Finally, it provides critical lenses through which analyse and judge design outcomes. We evaluated our tool in nine ideation workshops with a total of 32 participants. Results show that the tool was useful in informing and guiding idea generation and was perceived as appealing and fun. Drawing on observations and participant feedback, we reflect on the strengths and limitations of this tool.
\end{quote}

\emph{Published in}: Proceedings of the 2017 Conference on Designing Interactive Systems. DIS 2017. 587--598. Edinburgh, United Kingdom: ACM, (2017).

\emph{Description}: This paper introduces the Tiles ideation toolkit, a card based design and ideation toolkit for IoT, see Fig.~\ref{fig:framework} for an overview on how it connects with the Tiles IoT framework. The paper discusses the very first design iteration on the toolkit, its evaluation, strengths and limitations emerged. The Tiles ideation toolkit represents the first step towards improving participation and co-design (Chapter~\ref{cha:theory}) in the smart city learning domain.
The paper investigates RQ1 and RQ2, addressing how non-experts can be included in the development process of technological applications for the smart city domain.


\section[P3: Designing IoT Applications for Smart Cities: Extending the Tiles Ideation Toolkit.][Paper 3]{Paper 3}
\label{paper-3}

\emph{Title}: Designing IoT Applications for Smart Cities: Extending the Tiles Ideation Toolkit.

\emph{Authors}: Francesco Gianni and Monica Divitini.

\emph{Contributions of the authors}: Gianni designed and created the extension of the toolkit, new cards were designed and produced from scratch, several new groups of cards were created, existing cards were redesigned, a new redesigned cardboard was created and a new improved design process (playbook) was devised. Gianni also supervised the user study, collected and analysed the data and led the paper writing. Divitini provided general supervision for the design of the extension, research and the paper writing.

\begin{quote}
	\emph{Abstract:} The internet of things (IoT) is gaining momentum as a technical tool and solution for a diverse range of societal challenges. These challenges include smart cities sustainability issues which are widely recognised by decision makers and societies. Despite this, few works try to tackle these challenges empowering citizens through IoT technologies. In this paper we describe how the Tiles toolkit, a card based idea generation toolkit for IoT, has been extended to support non-experts in creating ideas addressing societal challenges that affect modern smart cities. We briefly introduce the Tiles generic toolkit, then we describe in detail the extensions proposed on the cards, cardboard and how the new components are employed in a refined workshop protocol. We report the results obtained during a field study of the extended toolkit, where several groups of students collaborated to generate ideas involving IoT in the smart city. We discuss success and failures, drawing our conclusions after analysing quantitative and qualitative data collected during the workshop. We conclude the article reporting the lessons learned, critical considerations about our experience evaluating the extended toolkit and reflections on possible improvements for future works.
\end{quote}

\emph{Published in}: Interaction Design and Architecture(s) Journal - IxD\&A 35:100--116, (2018).

\emph{Description}: This paper presents the second and final design iteration on the Tiles ideation toolkit. New cards and a new cardboard are introduced, the workshop protocol is refined to improve usability and support for unattended activities without facilitators. Compared to P2, the smart city learning domain is better integrated in the toolkit and in the workshop protocol.
With P3, the research work covering the first three phases of the Tiles IoT framework process (Fig.~\ref{fig:ideation-process}) is considered complete. The artefacts produced and the idea generation process involving non-experts were evaluated, leading to satisfactory results in terms of outcome of creative ideas, learning and support provided. The paper investigates RQ1 and RQ2, addressing how non-experts can generate ideas of IoT applications for smart cities, identifying a specific design strategy centred around tangible interfaces and augmented objects (Chapter~\ref{cha:theory}), and finally linking to the subsequent rapid prototyping phase.


\section[P4: Tiles-Reflection: Designing for Reflective Learning and Change Behaviour in the Smart City.][Paper 4]{Paper 4}
\label{paper-4}

\emph{Title}: Tiles-Reflection: Designing for Reflective Learning and Change Behaviour in the Smart City.

\emph{Authors}: Francesco Gianni, Lisa Klecha and Monica Divitini.

\emph{Contributions of the authors}: Gianni led the paper writing, Klecha designed the study and performed the theoretical grounding. Gianni and Klecha defined the extension of the toolkit, participated in the user studies, collected and analysed the data. Gianni performed the last iteration of the study, including the design and production of the cards for reflective learning and extended persona canvas. Gianni performed the user study, data collection and analysis. Divitini provided general supervision for the design of the extension, research and the paper writing.

\begin{quote}
	\emph{Abstract:} Modern cities are increasing in geographical size, population and number. While this development ascribes cities an important function, it also entails various challenges. Efficient urban mobility, energy saving, waste reduction and increased citizen participation in public life are some of the pressing challenges recognised by the United Nations. Retaining liveable cities necessitates a change in behaviour in the citizens, promoting sustainability and seeking an increase in the quality of life. Technology possesses the capabilities of mediating behaviour change. A review of existing works highlighted a rather unilateral utilisation of technology, mostly consisting of mobile devices, employment of persuasive strategies for guiding behaviour change, and late end-user involvement in the design of the application, primarily for testing purposes. These findings leave the door open to unexplored research approaches, including opportunities stemming from the Internet of Things, reflective learning as behaviour change strategy, and active involvement of end-users in the design and development process. We present Tiles-Reflection, an extension of the Tiles toolkit, a card-based ideation toolkit for the Internet of Things. The extension comprises components for reflective learning, allowing thus non-expert end-users to co-create behaviour change applications. The results of the evaluation suggest that the tool was perceived as useful by participants, fostering reflection on different aspects connected to societal challenges in the smart city. Furthermore, application ideas developed by the users successfully implemented the reflective learning model adopted.
\end{quote}

\emph{Published in}: The Interplay of Data, Technology, Place and People for Smart Learning. SLERD 2018. \emph{Smart Innovation, Systems and Technologies} 95:70--82. Aalborg, Denmark: Springer, Cham, (2019).

\emph{Description}: This paper presents an extension of the Tiles ideation toolkit which targets the brainstorming and design of IoT applications for behaviour change. Smart cities are used as domain, and CSRL in employed as behaviour change strategy. New cards and a new storyboard are used to guide the users in the ideation process of IoT application which support reflective learning. This work demonstrates how the Tiles ideation toolkit can be extended to support specific domain problems and learning strategies. The paper investigates RQ2 by introducing a different design goal compared to P3 and P2, namely behaviour change through reflective learning.


\section[P5: Designing IoT Applications in Lower Secondary Schools.][Paper 5]{Paper 5}
\label{paper-5}

\emph{Title}: Designing IoT Applications in Lower Secondary Schools.

\emph{Authors}: Anna Mavroudi, Monica Divitini, Francesco Gianni, Simone Mora and Dag R. Kvittem.

\emph{Contributions of the authors}: Mavroudi led the paper writing and the theoretical learning framing. Divitini provided general supervision for the design of the toolkit, research, and contributed in writing the paper. Gianni and Mora designed the study, the tools and performed the data collection and analysis. Kvittem provided facilitation and guidance during the user study. All the authors were present on-site during the user study.

\begin{quote}
	\emph{Abstract:} The paper reports on a case study where four groups of lower secondary school students participated in a workshop and undertook the demanding role of designers of Internet of Things applications. In doing that, they made use of a dedicated inventor toolkit, which facilitated students’ creative solutions to problems that can appear in the context of a smart city. From a pedagogical point of view, the workshop format is inline with the experiential learning approach. The paper presents a holistic student assessment methodology for this nice domain. In particular, to analyse the impact of the workshop for the students we used four different approaches: artefacts analysis of students’ design solutions, classroom observations, a post-test and a survey. The results indicate that the intervention has promoted an effective teaching methodology for the basic conceptual and design aspects of the IoT for these lower secondary school students, but it has not addressed equally effectively the attitude-related aspects. Nonetheless, the participant students perceived the intervention as very satisfactory in terms of the IoT concept knowledge, smart cities learning, and problem-solving skills acquired, as well as in terms of enjoyment. The paper concludes on the learning gains of the intervention and discusses the motivation aspect for the teacher as well as for the students in this highly innovative learning experience.
\end{quote}

\emph{Published in}: Proceedings of IEEE Global Engineering Education Conference. EDUCON 2018. 1120--1126. Tenerife, Spain: IEEE, (2018).

\emph{Description}: This paper evaluates the learning potential of the Tiles ideation toolkit in a classroom setting. Learning about the application domain, solution space and design methods are fundamental steps when targeting creative activities involving non-experts. The toolkit demonstrated to be effective in conveying knowledge about IoT, design, and in promoting smart city learning. Quantitative and qualitative data collected suggested the possibility to include the workshop as an integrated activity in the curriculum of the school.
The paper investigates RQ2, addressing how the toolkit can be employed for specific goals: learning about IoT, smart cities and design methods.


\section[P6: RapIoT Toolkit: Rapid Prototyping of Collaborative Internet of Things Applications.][Paper 6]{Paper 6}
\label{paper-6}

\emph{Title}: RapIoT Toolkit: Rapid Prototyping of Collaborative Internet of Things Applications.

\emph{Authors}: Francesco Gianni, Simone Mora and Monica Divitini.

\emph{Contributions of the authors}: Gianni led the paper writing, designed and supervised the development of the mobile application, designed and ran the evaluation study, collected and analysed the data and contributed in the development of the cloud based software and the electronic devices firmware. Mora contributed in writing the paper, designed the original architecture of the system, designed and produced the electronic devices. Both Gianni and Mora supervised the implementation of the system, and contributed in form of programming and testing. Divitini provided general supervision for research and paper writing.

\begin{quote}
	\emph{Abstract:} The Internet of Things holds huge promise in enhancing collaboration in multiple application domains. Bringing internet connectivity to everyday objects and environments promotes ubiquitous access to information and integration with third-party systems. Further, connected \enquote{things} can be used as physical interfaces to enable users to cooperate, leveraging multiple devices via parallel and distributed actions. Yet creating prototypes of IoT systems is a complex task for developers non-expert in IoT, as it requires dealing with multi-layered hardware and software infrastructures. We introduce RapIoT, a software toolkit that facilitates the prototyping of IoT systems by providing an integrated set of technologies. Our solution abstracts low-level details and communication protocols, allowing developers non-expert in IoT to focus on application logic, facilitating rapid prototyping. RapIoT supports the development of collaborative applications by enabling the definition of high-level data type primitives and allowing interactions spread among multiple smart objects. RapIoT primitives act as a loosely coupled interface between generic IoT devices and applications, simplifying the development of systems that make use of an ecology of devices distributed to multiple users and environments. We illustrate the potential of our toolkit by presenting the development process of an IoT application ideated during a workshop with non-expert developers and addressing real-world challenges affecting smart cities. We conclude by discussing the strength and limitations of our platform, highlighting further possible uses for collaborative applications.
\end{quote}

\emph{Published in}: Journal of Future Generation Computer Systems. Elsevier, (2018).

\emph{Description}: This paper presents RapIoT, a toolkit to support rapid prototyping of IoT applications, see Fig.~\ref{fig:framework} for an overview on how it connects with the Tiles IoT framework. The toolkit is composed by software to be deployed on the cloud, on mobile devices and on microcontrollers contained in embedded devices. It is designed having non-experts in mind, thus avoiding intricate deployment procedures and complex programming strategies. An event-based messaging protocol connects the electronic devices deployed on the field with the application logic running in the cloud. These messages are human readable and present a simple structure, thus allowing debugging and extensions with limited effort.
The paper investigates RQ1 and RQ3, addressing the prototyping phase of IoT applications and how technology can support it best when non-experts are involved.


\section[P7: Rapid Prototyping Internet of Things Applications for Augmented Objects: The Tiles Toolkit Approach.][Paper 7]{Paper 7}
\label{paper-7}

\emph{Title}: Rapid Prototyping Internet of Things Applications for Augmented Objects: The Tiles Toolkit Approach.

\emph{Authors}: Francesco Gianni, Simone Mora and Monica Divitini.

\emph{Contributions of the authors}: Gianni wrote the paper, designed and ran the evaluation study, collected and analysed the data, designed and produced one of the electronic devices and programmed both of them. Mora contributed in writing the paper, designing and producing the second electronic device, and in running the evaluation study. Divitini provided general supervision for research and paper writing.

\begin{quote}
	\emph{Abstract:} Designing and prototyping for IoT have historically required a diverse range of skills and a set of tools that individually supported only a fraction of the whole process, not being designed to work together. These tools usually require a certain level of proficiency in design methods, programming or electronics, depending on the phase addressed. Previous works on the Tiles Ideation toolkit and the RapIoT software framework demonstrated how the design phase can be democratised and how a simple programming paradigm can make coding for IoT a task accessible to non-experts. With this work we present and evaluate the process and the technologies involved in the programming and prototyping phase of an IoT application. The Tiles Square and the Tiles Temp are introduced, these two electronic devices complement and support IoT prototyping. They are designed to work in conjunction with the Tiles Ideation toolkit and are supported by the RapIoT software framework, allowing non-experts to augment and program everyday objects. We illustrate the potential of this approach by presenting the results obtained after workshops with 44 students. We conclude by discussing strengths and limitations of our approach, highlighting the lessons learned and possible improvements.
\end{quote}

\emph{Published in}: Ambient Intelligence. AmI 2018. \emph{Lectures Notes in Computer Science.} 11249:204--220. Larnaca, Cyprus: Springer, Cham, (2018).

\emph{Description}: This paper evaluates the rapid prototyping of IoT application ideas generated with the Tiles ideation toolkit. Two electronic devices were designed and built, allowing non-experts to augment objects by simply attaching them to things. The RapIoT toolkit is used to support the users in programming the application behaviour. RapIoT also provided network connectivity to the electronic devices attached to the objects, and assisted the users in the initial physical deployment procedure. The evaluation demonstrated how non-experts were able to complete the steps required to transform an application idea into a tangible prototype, embedding into an object the application behaviour envisioned. Such outcome also validates the transition between the idea generation and design phases, where the Tiles ideation toolkit is employed, and the prototyping phase.
The paper contributes in answering all the research questions. It addresses mainly the prototyping phase (RQ3), but it is strongly tied to the preceding articles, building on the assumptions, boundaries and design choices that drove the entire development process.

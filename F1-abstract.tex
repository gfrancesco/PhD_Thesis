\chapter{Abstract}

Interacting with computers, for work or for leisure, originally required humans to \textit{speak the language of the machine}, intended as interaction techniques, languages and procedures that are closer to how a computer works rather than how the human brain functions. Nowadays, despite the wide availability of computers, this gap is still present. It is now easier to use computers; however, programming even simple applications is still a task approachable only through complex jargon, which has little in common with the final scope of the application.
By addressing the Internet of Things (IoT), we found that this situation is complicated by the involvement of electronics, microcontrollers and low-level programming languages. Specific professional skills are often required to develop and prototype IoT applications.
Research on human-computer interaction (HCI) addresses the challenges of interconnecting people and computers, building tools and theories to facilitate the many uses that a computer can have, in an open-ended dialogue. Thanks to research in this field, new solutions and interaction strategies allowed us to improve user experience when dealing with computers. However, owing to the complexity of the field, in IoT, it is still challenging to keep humans in the loop, in terms of both the development of IoT applications and their use. Specific branches in HCI aim to facilitate the programming phase of computer applications; however, despite being simple, such process still requires the user to have some non-trivial technical skills and an understanding of the basic logical constructs common to most programming languages.

In this thesis, I address how to empower new audiences in brainstorming, designing, programming and prototyping applications for IoT. Already established research fields, such as end-user development (EUD), HCI, interaction design (ID) and software engineering, have already investigated some of these challenges. In this thesis, I will restrict the domain to a specific subset of IoT applications based on tangible user interfaces (TUIs) and smart objects, covering the phases of brainstorming and design of such applications. Particular focus will be placed on promoting smart city learning (SCL) through the IoT applications envisioned. SCL explores how citizens can be actively involved in a learning process that occurs in the city, making use of its data and services, in order to increase awareness and lifelong learning. The SCL application domain was chosen since it can benefit from IoT technologies, promoting inclusion and participation. In fact, IoT is scarcely used in applications for SCL, while citizens struggle to contribute actively in the life of the city. In order to achieve this inclusive vision, the entry barriers to technology need to be lowered and an integrated process guiding the users during all the phases of the development process should be devised. In this thesis, I will explore how concepts from design thinking, HCI and user-centred design can be implemented in novel tools and processes to support such goal.

This work is grounded in design science research methodologies. Several prototypes were built during eight design iterations, and field studies have been performed at the end of each iteration, for evaluation and validation against acceptance, usability and impact on the problem at stake. Software and hardware rapid prototyping techniques and open-source and digital manufacturing tools have been largely employed. The results of the evaluations were used for validating the already existing theories in HCI, SCL and IoT and for the development of new constructs. Commercial exploitation of the research outcomes is underway.

The resulting contributions add new knowledge to guide the ideation, design, programming and prototyping of IoT applications for users with limited technical skills, placing particular emphasis on the case of applications for SCL. A holistic framework for IoT has been devised to bootstrap the ideation process and guide the users towards a tangible, programmable prototype of the application idea. The work described in this thesis includes the implementation of two toolkits for ideation, programming and physical prototyping. This development task required a wide range of competencies in design, software and hardware engineering. Some of these critical competencies were acquired by the author during the course of the investigation. The lessons learned from the experience of the author provided knowledge connected to the creation of tools to ease idea generation and rapid prototyping of IoT applications.
